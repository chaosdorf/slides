\documentclass{beamer}
\usepackage[utf8]{inputenc}
\usepackage{listings}
\usepackage[orientation=landscape,size=custom,width=16,height=9,scale=0.75]{beamerposter}
\usetheme{Warsaw}
\title[Der Chaosdorf-Adminreport]
	{Der Chaosdorf-Adminreport}
\author{Daniel Friesel \and Maximilian Gaß}
\institute{Chaosdorf}
\date{1. Oktober 2010}

\renewcommand{\familydefault}{\sfdefault}

\lstset{numbers=left}

\begin{document}

\begin{frame}
	\titlepage
\end{frame}


\begin{frame}{Am Anfang war das FreeBSD}
	\begin{itemize}
		\item FreeBSD-System mit Jails auf chaosdorf.de
		\item Von ? bis Sommer 2009
	\end{itemize}
\end{frame}

\begin{frame}{Anforderungen}
	\begin{itemize}
		\item Zentrale Nutzerverwaltung
		\item Shellserver
		\item Mailserver
		\item Website
		\item Internes Wiki + Interner Bugtracker
	\end{itemize}
\end{frame}

\section{Systembeschreibung}

\begin{frame}{The Next Generation}
	\begin{itemize}
		\item To Boldly Serve How Noone Has Served Before
	\end{itemize}
	\begin{itemize}
		\item vm.chaosdorf.de
		\begin{itemize}
			\item backend.chaosdorf.de
			\item frontend.chaosdorf.de
			\item shells.chaosdorf.de
		\end{itemize}
	\end{itemize}
	\begin{itemize}
		\item Debian Stable
	\end{itemize}
\end{frame}


\subsection{vm.chaosdorf.de}
\begin{frame}{Hardware und Hosting}
	\begin{itemize}
		\item Hosting by OpenIT
		\item 217.69.82.240/29 in eigenem VLAN
		\item Keinerlei Remote-Zugriff außer SSH
	\end{itemize}
\end{frame}

\begin{frame}{vm.chaosdorf.de}
	\begin{itemize}
		\item Drei VMs mit qemu-kvm + libvirt
		\item Zwei Festplatten, RAID 1 + LVM
		\item Verschlüsselte Volumes für die einzelnen VMs
		\item Paketfilter mit ferm, auch zwischen VMs etc.
	\end{itemize}
\end{frame}

\subsection{backend.chaosdorf.de}
\begin{frame}{backend.chaosdorf.de}
	\begin{itemize}
		\item LDAP
		\item (Noch:) Webserver mit phpldapadmin
		\item approx (Debian-Paketcache)
		\item reprepro für das Admin-Toolkit
	\end{itemize}
\end{frame}

\subsection{frontend.chaosdorf.de}
\begin{frame}{frontend.chaosdorf.de}
	\begin{itemize}
		\item Mail
		\begin{itemize}
			\item Postfix mit postgrey und SpamPD
			\item Mailinglisten (miniml)
			\item Roundcube mit MySQL
			\item Dovecot
			\item Alle mit LDAP-Anbindung
		\end{itemize}
	\end{itemize}
\end{frame}

\begin{frame}{frontend.chaosdorf.de}
	\begin{itemize}
		\item Webserver: Apache2
		\begin{itemize}
			\item ikiwiki: Website und Wiki
			\item Ticketsystem: Roundup mit SQLite
		\end{itemize}
		\item Userzugriff: SSH für git + SFTP für Userwebsites
		\item Userquotas
	\end{itemize}
\end{frame}

\subsection{shells.chaosdorf.de}
\begin{frame}{shells.chaosdorf.de}
	\begin{itemize}
		\item Shell-Zugriff. Sonst Nichts.
		\item Userlimits und Quotas
	\end{itemize}
\end{frame}

\section{Clubraum-Infrastruktur}
\subsection{Router}
\begin{frame}{Clubraum-Infrastruktur}
	\begin{itemize}
		\item Raumserver: figurehead
		\item Routing mit dn42-Anschluss
		\item MPD
		\item FTP
		\item In Zukunft: WLAN-AP
		\item Keine Useraccounts, daher keine LDAP-Anbindung
	\end{itemize}
\end{frame}

\subsection{Terminal}
\begin{frame}{Clubraum-Terminal}
	\begin{itemize}
		\item In Planung
		\item Mit LDAP-Anbindung
		\item Option, eigene Livesysteme vom USB-Stick zu booten
		\item Problem: Booten externer Sticks vs. LDAP-Passwort auf
			Festplatte
	\end{itemize}
\end{frame}

\section{Weitere Themen}

\subsection{Metafoo}
\begin{frame}{Dokumentation und Transparenz}
	\begin{itemize}
		\item Doku schreiben, während man das System aufsetzt. Nicht später
		\item Änderungen: im öffentlichen Log + in \#chaosdorf
		\item Idee: Auditing mit sudo und Logcheck
		\item User haben Lesezugriff auf's Monitoring
		\item Alle Änderungen werden angekündigt
	\end{itemize}
\end{frame}

\begin{frame}{Versionskontrolle}
	\begin{itemize}
		\item etckeeper (Git für /etc)
		\item Ebenfalls: Git für /usr/local
	\end{itemize}
\end{frame}

\subsection{Monitoring}
\begin{frame}{Monitoring}
	\begin{itemize}
		\item Will man haben
		\item Auf vom System komplett unabhängigen Host
		\item Zusätzliche Checks schaden nie
		\item Für SSH-CHecks: Separater key mit forcecommand
		\item Goodie: Ankündigungen im IRC per Bot
	\end{itemize}
\end{frame}

\begin{frame}{Besondere Checks}
	\begin{itemize}
		\item Git-Status (/etc + /usr/local)
		\item Kein offener Mailrelay
		\item Kein SSH-Passwortlogin
		\item RBL (frontend nicht blacklisted)
		\item Laufende Kernelversion
		\item SSL-Zertifikate
	\end{itemize}
\end{frame}

\subsection{Automatisierung}
\begin{frame}{Automatisierung}
	\begin{itemize}
		\item Puppet (Rubyrant von mxey)
		\item $\rightarrow$ Admin-Toolkit als .deb mit reprepro
	\end{itemize}
\end{frame}

\begin{frame}{Siehe auch}
	\begin{itemize}
		\item https://intern.chaosdorf.de/admin/
		\item http://github.com/chaosdorf/chaosdorf-admin-toolkit
	\end{itemize}
\end{frame}

\end{document}
