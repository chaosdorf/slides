\documentclass{beamer}
\usepackage[utf8]{inputenc}
\usepackage{listings}
\usetheme{Warsaw}
\title[Keep It Simple and Stupid]
	{Keep It Simple and Stupid\\Das KISS-Prinzip in der Praxis}
\author{Daniel Friesel \and Maximilian Gaß}
\institute{Chaosdorf}
\date{30. April 2010}

\renewcommand{\familydefault}{\sfdefault}

\lstset{numbers=left}

\begin{document}

\begin{frame}
	\titlepage
\end{frame}

\begin{frame}{Was ist KISS?}
	\begin{itemize}
		\item KISS -- Keep It Simple and Stupid
		\item (auch: Keep It Simple, Stupid)
		\pause \item Es geht nicht primär um Bloat oder Performancevorteile
	\end{itemize}
	\pause
	\begin{quote}
		Quick C ist nicht Quick, SMTP ist nicht Simple, LDAP ist nicht
		Lightweight, SQL ist nicht Structured, Windows 2000 Professional
		ist
		nicht Professional.  Seht ihr?  Ist doch ganz einfach.
	\end{quote}
	\begin{flushright}
		-- Felix von Leitner, de.comp.security.misc
	\end{flushright}
\end{frame}
% z.B. goblin (plan9-userland) - "If in doubt, use brute force" (Naiver
% Algorithmus ist einfacher zu lesen)

\section{Initskripte}

\begin{frame}{Initskripte}
	Was sind Initskripte?\\

	\begin{itemize}
		\pause \item Starten/Stoppen Services (Daemons)
		\pause \item Integraler Bestandteil jedes unixoiden Systems
	\end{itemize}
\end{frame}

\subsection{Debian}
\begin{frame}[allowframebreaks, allowdisplaybreaks]{Debian}
	\lstset{basicstyle=\tiny}
	\lstinputlisting{mpd/debian}
\end{frame}

\subsection{Arch Linux}
\begin{frame}[allowframebreaks, allowdisplaybreaks]{Arch Linux}
	\lstset{basicstyle=\tiny}
	\lstinputlisting{mpd/arch}
\end{frame}

\subsection{daemontools}
\begin{frame}{daemontools}
	\lstinputlisting{mpd/daemontools}
\end{frame}

% upstart ist noch nicht so ausgereift.
% Dafür hat Simplicity in diesem Bereich keine Nachteile.
\subsection{upstart}
\begin{frame}{upstart}
	\lstinputlisting{mpd/upstart}
\end{frame}

\subsection{Fazit}
\begin{frame}{Fazit}
	\begin{itemize}
		\item Gewaltige Unterschiede
		\pause \item Simplicity hat in diesem Bereich keine Nachteile
	\end{itemize}
\end{frame}

\section{Linking}
\begin{frame}{Was ist Linking?}
	\begin{itemize}
		\item Einbinden von Modulen in (C-)Programme
		\pause \item Statisches vs. Dynamisches Linken
	\end{itemize}
\end{frame}

\subsection{Statisches Linken}
\begin{frame}{Statisches Linken}
	\begin{itemize}
		\item Kein Linken beim Programmstart
		\pause \item Keine Probleme mit ABI-Changes
		\pause \item Weniger Abhängigkeiten auf Libraries
	\end{itemize}
\end{frame}

\subsection{Dynamisches Linken}
\begin{frame}{Dynamisches Linken}
	\begin{itemize}
		\item Ein zentraler Patch fixt alle Binaries
		\pause \item Geringerer RAM-Verbrauch und Plattenplatzbedarf
		\pause \item Standard, daher wenig fummelig beim Bauen
	\end{itemize}
\end{frame}

\subsection{Fazit}
\begin{frame}{Fazit}
	\begin{itemize}
		\item Situationsabhängig, aber dynamisches Linken ist de-facto
			Standard
	\end{itemize}
\end{frame}

\section{zsh / mksh}
\begin{frame}{Zsh vs. mksh}
	\begin{itemize}
		\item zsh: Z Shell, extrem konfigurierbar und featurereich
		\item mksh / pdksh: Korn Shell-Varianten, können einiges, aber weniger
	\end{itemize}
\end{frame}

\subsection{Nachteile der zsh}
\begin{frame}{Nachteile der zsh}
	\begin{itemize}
		\item Sehr groß (536KiB Binary, mksh/pdksh: 224/196 KiB)
		\pause \item Hoher RAM-Verbrauch (5-10M pro Instanz)
		\pause \item Sehr komplex % (v.a. Completion, kann hängen)
	\end{itemize}
	\pause
	\begin{quote}
		shell programming is like LSD, you often see advantages that are not
		there.
	\end{quote}
	\begin{flushright}
		-- Zaba
	\end{flushright}
\end{frame}

\subsection{Vorteile der mksh/pdksh}
\begin{frame}{Vorteile der mksh/pdksh}
	\begin{itemize}
		\item Niedriger RAM-Verbrauch (500-800K)
		\pause \item Schnelle Startup-Zeit % kein compinit
		\pause \item Verhältnismäßig simpel
	\end{itemize}
\end{frame}

\subsection{Bequemlichkeit}
\begin{frame}{Aber: Bequemlichkeit}
	\begin{itemize}
		\item zsh kann einfach mehr
		\pause \item $\rightarrow$ Abwägen, was einem wichtiger ist
	\end{itemize}
\end{frame}

\section{hal}
\begin{frame}{HAL}
	\begin{itemize}
		\item ``Hardware Abstraction Layer''
		\pause \item Input-Hotplug, wobei X einzelne Geräte sehen kann
		\pause \item Hotplug selbst ging und geht auch ohne HAL
		\pause \item Klassisch: Flexibilität für wenige $\rightarrow$ mehr
			Komplexität für alle
		\pause \item Obskure XML-Configs
		\pause \item Inzwischen: libudev
	\end{itemize}
\end{frame}

\section{udev / mdev}
\begin{frame}{udev vs. mdev}
	Device-Hotplugging (/dev)
\end{frame}

\subsection{udev}
\begin{frame}{udev}
	\begin{itemize}
		\item Daemon (Sicherheitsprobleme)
		\pause \item udevadm settle: Teilweise sehr langsam
	\end{itemize}
\end{frame}

\subsection{mdev}
\begin{frame}{mdev}
	\begin{itemize}
		\item Kein Daemon
		\pause \item kein Netlink-Socket, klassische
			Argumente / Environment
		\pause \item Hat trotzdem fast alle udev-Features
		\pause \item Schnell (~0.1s für alle Devicenodes)
	\end{itemize}
\end{frame}

\section{Abschluss}

\begin{frame}{Fazit}
	\begin{quote}
		Linux is becoming an idiot box; a nanny operating system. The
		complexity and
		the idiot box ideals are killing the power user, and the spare time
		developer.
	\end{quote}
	\begin{flushright}
		-- Tuomo Valkonen
	\end{flushright}
\end{frame}

\begin{frame}{Links}
	\begin{itemize}
		\item suckless.org
		\item harmful.cat-v.org
	\end{itemize}
\end{frame}

\end{document}
